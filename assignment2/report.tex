\documentclass{article}
\usepackage{geometry}
\usepackage{graphicx}
\usepackage{amsmath}
\usepackage{algorithm}
\usepackage{algpseudocode}
\usepackage{multicol}
\usepackage{multirow}
\usepackage{arydshln}
\usepackage[usenames, dvipsnames]{color}
\geometry{
a4paper,
right=10mm,
left=10mm,
top=5mm,
bottom=5mm,	
}
\newcommand\tab[1][5pt]{\hspace*{#1}} 
\newcommand\sep{\tab | \tab } 
\begin{document}

\pagenumbering{gobble}

\begin{center}
\textbf{\huge Assignment 2 : CS335} \\
\textit{\Large Jayant Agrawal}         14282 \\
Section S1
\end{center}

\textbf{\huge Parsing}
\section*{Sol1: Ambiguous Grammar?}
\textbf{(a).} This is a post fix grammer which produces post fix expressions. \\ \\
\textbf{(b).} Since, the SLR parse table(Figure \ref{slr1}) has no conflict, this grammer is an SLR grammer and thus is unambiguous.
The grammer: $G$ has the following productions(numbered):
\begin{enumerate}
\item $S \rightarrow S S +$
\item $S \rightarrow S S *$
\item $S \rightarrow a$
\end{enumerate}
Adding the production $S' \rightarrow S$ to get augmented grammer $G'$ and computing \emph{Sets-of-items}(Figure \ref{soi1}):
\begin{figure}[h!]
\begin{multicols}{2}
\begin{equation*}
\begin{aligned}
I_0 &: S' \rightarrow .S  \\
&: S \rightarrow .SS+ \\
&: S \rightarrow .SS* \\
&: S \rightarrow .a \\
\end{aligned}
\end{equation*}

\begin{equation*}
\begin{aligned}
I_2 : goto(I_0,a) &: S \rightarrow a.  \\
\end{aligned}
\end{equation*}

\begin{equation*}
\begin{aligned}
I_1 : goto(I_0,S) &: S \rightarrow S.S+  \\
&: S \rightarrow S.S*  \\
&: S \rightarrow .SS+  \\
&: S \rightarrow .SS*  \\
&: S \rightarrow .a  \\
\end{aligned}
\end{equation*}

\begin{equation*}
\begin{aligned}
I_3 : goto(I_1,S) &: S \rightarrow SS.+  \\
&: S \rightarrow SS.*  \\
&: S \rightarrow S.S+  \\
&: S \rightarrow S.S*  \\
&: S \rightarrow .SS+  \\
&: S \rightarrow .SS*  \\
&: S \rightarrow .a  \\
\end{aligned}
\end{equation*}

\begin{equation*}
\begin{aligned}
I_4 : goto(I_3,+) &: S \rightarrow SS+.  \\
\end{aligned}
\end{equation*}

\begin{equation*}
\begin{aligned}
I_5 : goto(I_3,*) &: S \rightarrow SS*.  \\
\end{aligned}
\end{equation*}

\begin{equation*}
\begin{aligned}
I_2 &: goto(I_1,a)  \\
I_2 &: goto(I_3,a)  \\
I_3 &: goto(I_3,S)  \\
\end{aligned}
\end{equation*}

\end{multicols}
\caption{Sol1: Sets-of-items}
\label{soi1}
\end{figure}

\begin{figure}[h!]
\begin{center}
\begin{tabular}{||c | c c c c  | c ||}
\hline
\hline
\multirow{2}{*}{\textbf{STATE}} & \multicolumn{4}{c|}{\emph{action}}  & \multicolumn{1}{c||}{\emph{goto}} \\
\cline{2-6}
& \bf $\ast$ & \bf + & \bf a & \$ & \bf S  \\
\hline
0 & & & {\color{ForestGreen} s2} &  & 1 \\
\hdashline[0.5pt/10pt]
1 & & & {\color{ForestGreen} s2} & {\color{red} \emph{acc}} & 3 \\
\hdashline[0.5pt/10pt]
2 & {\color{blue} r3} &{\color{blue} r3} & {\color{blue} r3} & {\color{blue} r3} &  \\
\hdashline[0.5pt/10pt]
3 & {\color{ForestGreen} s5}&{\color{ForestGreen} s4} & {\color{ForestGreen} s2} &  & 3 \\
\hdashline[0.5pt/10pt]
4 & {\color{blue} r1} &{\color{blue} r1} & {\color{blue} r1} & {\color{blue} r1} &  \\
\hdashline[0.5pt/10pt]
5 & {\color{blue} r2} &{\color{blue} r2} & {\color{blue} r2} & {\color{blue} r2} &  \\
\hline
\hline
\end{tabular}
\end{center}
\caption{Sol1: SLR Parse Table}
\label{slr1}
\end{figure} 

\newpage
\section*{Sol2: Parse Table}
The grammer: $G$ has the following productions(numbered):
\begin{enumerate}
\item $S \rightarrow Ma$
\item $S \rightarrow bMc$
\item $S \rightarrow dc$
\item $S \rightarrow bda$
\item $M \rightarrow d$
\end{enumerate}
Adding the production $S' \rightarrow S$ to get augmented grammer $G'$ and computing \emph{Sets-of-items}(Figure \ref{soi}):
\begin{figure}[h!]
\begin{multicols}{2}
\begin{equation*}
\begin{aligned}
I_0 &: S' \rightarrow .S  \\
&: S \rightarrow .Ma \hspace{5pt} | \tab .bMc \tab | \tab .dc \tab | \tab .bda \\
&: M \rightarrow .d
\end{aligned}
\end{equation*}

\begin{equation*}
\begin{aligned}
I_1 : goto(I_0,S) &: S' \rightarrow S.  \\
\end{aligned}
\end{equation*}


\begin{equation*}
\begin{aligned}
I_2 : goto(I_0,M) &: S \rightarrow M.a  \\
\end{aligned}
\end{equation*}


\begin{equation*}
\begin{aligned}
I_3 : goto(I_2,a) &: S \rightarrow Ma.  \\
\end{aligned}
\end{equation*}


\begin{equation*}
\begin{aligned}
I_4 : goto(I_0,b) &: S \rightarrow b.Mc \tab | \tab b.da  \\
&: M \rightarrow .d \\
\end{aligned}
\end{equation*}


\begin{equation*}
\begin{aligned}
I_5 : goto(I_4,M) &: S \rightarrow bM.c  \\
\end{aligned}
\end{equation*}


\begin{equation*}
\begin{aligned}
I_6 : goto(I_4,d) &: S \rightarrow bd.a  \\
&: M \rightarrow d.  \\
\end{aligned}
\end{equation*}

\begin{equation*}
\begin{aligned}
I_7 : goto(I_5,c) &: S \rightarrow bMc.  \\
\end{aligned}
\end{equation*}

\begin{equation*}
\begin{aligned}
I_8 : goto(I_6,a) &: S \rightarrow bda.  \\
\end{aligned}
\end{equation*}

\begin{equation*}
\begin{aligned}
I_9 : goto(I_0,d) &: S \rightarrow d.c  \\
&: M \rightarrow d.  \\
\end{aligned}
\end{equation*}


\begin{equation*}
\begin{aligned}
I_{10} : goto(I_9,c) &: S \rightarrow dc.  \\
\end{aligned}
\end{equation*}
\end{multicols}
\caption{Sol2: Sets-of-items}
\label{soi}
\end{figure}

Also, the first and follow sets of the non-terminals in $G$ are as follows:\\
first(S) = \{b,d\}\\
first(M) = \{d\} \\
follow(S) = \{\$\} \\
follow(M) = \{a,c\} \\ \\
Using the above first, follow sets and the sets-of-items, we get the following SLR Parse Table (Figure \ref{slr} ) :
\begin{figure}[h!]
\begin{center}
\begin{tabular}{||c | c c c c c | c c||}
\hline
\hline
\multirow{2}{*}{\textbf{STATE}} & \multicolumn{5}{c|}{\emph{action}}  & \multicolumn{2}{c||}{\emph{goto}} \\
\cline{2-8}
& \bf a & \bf b & \bf c & \bf d & \$ & \bf S &\bf M \\
\hline
0 & & {\color{ForestGreen} s4} & & {\color{ForestGreen} s9} & & 1 &2 \\
\hdashline[0.5pt/10pt]
1 & & & & & {\color{red} \emph{acc}} &  & \\
\hdashline[0.5pt/10pt]
2 &{\color{ForestGreen} s3} &  & &  & &  & \\
\hdashline[0.5pt/10pt]
3 & & & & & {\color{blue} r1} &  & \\
\hdashline[0.5pt/10pt]
4 & & & & {\color{ForestGreen} s6} &  &  & 5\\
\hdashline[0.5pt/10pt]
5 & & & {\color{ForestGreen} s7}& &  &  & \\
\hdashline[0.5pt/10pt]
6 & \begin{tabular}{c} {\color{ForestGreen} s8} \\ {\color{blue} r5} \end{tabular} & & {\color{blue} r5}& &  &  & \\
\hdashline[0.5pt/10pt]
7 &  & & & & {\color{blue} r5} &  & \\
\hdashline[0.5pt/10pt]
8 &  & & & & {\color{blue} r4} &  & \\
\hdashline[0.5pt/10pt]
9 & {\color{blue} r5} & &\begin{tabular}{c} {\color{ForestGreen} s10} \\ {\color{blue} r5} \end{tabular} & &  &  & \\
\hdashline[0.5pt/10pt]
10 &  & & & & {\color{blue} r5} &  & \\
\hline
\hline
\end{tabular}
\end{center}
\caption{Sol2: SLR Parse Table}
\label{slr}
\end{figure}
\newpage
\textbf{(a).} There are two conflicts in the parse table(Figure \ref{slr}) : \emph{6a} and \emph{9c}. \\ \\

\textbf{(b).} $I_6$ and $I_9$ cause conflicts in Figure \ref{soi}. \\ \\

\textbf{(c).} The viable prefixes for the conflicting states are 'bd' and 'd'. \\ \\

\textbf{(d).} The conflicts are shift-reduce conflicts. Since, the knowledge of input is not available, there are two possible actions. If we had used LR(1), then we would have known the next input and thus would have made the choice based on that.
\section*{Sol3: SLR vs CLR vs LALR}
Consider the Sets-of-items (Figure \ref{soi3}) and Automation(Figure \ref{auto3}) for the given grammer.
\begin{figure}[h!]
\begin{multicols}{2}
\begin{equation*}
\begin{aligned}
I_0 &: S' \rightarrow .S  \\
&: S \rightarrow .id[E] := E
\end{aligned}
\end{equation*}

\begin{equation*}
\begin{aligned}
I_1 : goto(I_0,id) &: S \rightarrow id.[E] :=E  \\
\end{aligned}
\end{equation*}

\begin{equation*}
\begin{aligned}
I_2 : goto(I_0,S) &: S' \rightarrow S.  \\
\end{aligned}
\end{equation*}

\begin{equation*}
\begin{aligned}
I_3 : goto(I_1,[) &: S \rightarrow id[.E] :=E  \\
&: E \rightarrow .E+T \tab | \tab .T \\
&: T \rightarrow .T*F \tab | \tab  .F \\
&: F \rightarrow .(E) \sep .id \\
\end{aligned}
\end{equation*}

\begin{equation*}
\begin{aligned}
I_4 : goto(I_3,() &: F \rightarrow (.E)  \\
&: E \rightarrow .E+T \tab | \tab .T \\
&: T \rightarrow .T*F \tab | \tab  .F \\
&: F \rightarrow .(E) \sep .id \\
\end{aligned}
\end{equation*}

\begin{equation*}
\begin{aligned}
I_5 : goto(I_3,id) &: F \rightarrow id.  \\
\end{aligned}
\end{equation*}

\begin{equation*}
\begin{aligned}
I_6 : goto(I_3,E) &: S \rightarrow id[E.] :=E  \\
&: E \rightarrow E.+T \\
\end{aligned}
\end{equation*}

\begin{equation*}
\begin{aligned}
I_7 : goto(I_6,]) &: S \rightarrow id[E]. :=E  \\
\end{aligned}
\end{equation*}

\begin{equation*}
\begin{aligned}
I_8 : goto(I_6,+) &: E \rightarrow E+.T  \\
&: T \rightarrow .T*F \tab | \tab  .F \\
&: F \rightarrow .(E) \sep .id \\
\end{aligned}
\end{equation*}


\begin{equation*}
\begin{aligned}
I_9 : goto(I_3,T) &: E \rightarrow T.  \\
&: T \rightarrow T.*F  \\
\end{aligned}
\end{equation*}

\begin{equation*}
\begin{aligned}
I_{10} : goto(I_9,*) &: T \rightarrow T*.F  \\
&: F \rightarrow .(E) \sep .id
\end{aligned}
\end{equation*}


\begin{equation*}
\begin{aligned}
I_{11} : goto(I_3,F) &: T \rightarrow F.  \\
\end{aligned}
\end{equation*}


\begin{equation*}
\begin{aligned}
I_{12} : goto(I_4,E) &: F \rightarrow (E.)  \\
&: E \rightarrow E.+T 
\end{aligned}
\end{equation*}


\begin{equation*}
\begin{aligned}
I_{13} : goto(I_{12},)) &: F \rightarrow (E).  \\
\end{aligned}
\end{equation*}


\begin{equation*}
\begin{aligned}
I_{14} : goto(I_8,T) &: E \rightarrow E+T.  \\
&: T \rightarrow T.*F
\end{aligned}
\end{equation*}

\begin{equation*}
\begin{aligned}
I_{15} : goto(I_7,:=) &: S \rightarrow id[E] := .E  \\
&: E \rightarrow .E+T \tab | \tab .T \\
&: T \rightarrow .T*F \tab | \tab  .F \\
&: F \rightarrow .(E) \sep .id \\
\end{aligned}
\end{equation*}

\begin{equation*}
\begin{aligned}
I_{16} : goto(I_{10},F) &: T \rightarrow T*F.
\end{aligned}
\end{equation*}


\begin{equation*}
\begin{aligned}
I_{17} : goto(I_{15},E) &: S \rightarrow id[E] := E. \\
&: E \rightarrow E.+T \\
\end{aligned}
\end{equation*}

\begin{equation*}
\begin{aligned}
I_{9} &: goto(I_4,T) \\
I_8 &: goto(I_{12},+) \\
\color{red} {I_4} &: \color{red} {goto(I_{4},( \tab )} \\
I_5 &: goto(I_{4},id) \\
I_{11} &: goto(I_{4},F) \\
I_{10} &: goto(I_{14},*) \\
I_{11} &: goto(I_{8},F) \\ 
I_{4} &: goto(I_{8},C) \\
I_{10} &: goto(I_{9},*) \\
I_{5} &: goto(I_{8},id) \\
I_{4} &: goto(I_{10},( \tab ) \\
I_{5} &: goto(I_{15},id) \\
I_{9} &: goto(I_{15},T) \\
I_{11} &: goto(I_{15},F) \\
I_{4} &: goto(I_{15},( \tab ) \\
I_{8} &: goto(I_{17},+) \\
I_{5} &: goto(I_{10},id) \\
\end{aligned}
\end{equation*}
\end{multicols}
\caption{Sol3: LR(0) Sets-of-items }
\label{soi3}
\end{figure}

\newpage
\begin{figure}[h!]
\begin{center}
\includegraphics[scale=0.5]{autoSLR.png}
\caption{Sol3: Automation SLR }
\label{auto3}
\end{center}
\end{figure}

\newpage
\textbf{(a).} The set of viables prefixes are simply all the paths that lead to $I_4$(the state with a self loop) in Figure \ref{auto3} which are:

$$id[(^+ P $$
$$id[ E + (^+ P$$
$$id[ T \ast (^+ P$$
$$id[ E + T \ast (^+ P$$
$$id[E] :=  (^+ P$$
$$id[E] :=  E + (^+ P$$
$$id[E] :=  T \ast (^+ P$$

where $P =\{ \Big(T \ast (^+ \Big)\tab  | \tab \Big(E + (^+ \Big) \}^*$. \\ \\
\textbf{(b).} Consider the example string x =$id_1 [\tab ( \tab id_2 ] := id_3$ . Notice that, x has an unbalanced set of paranthesis, this it is not part of this grammer. Consider the behaviour of SLR parser made from the above items (Figure \ref{soi3}) on x. After $id_2$ is shifted from input to the stack, the parser would be in state 5 ($0-> 1-> 3-> 4-> 5$, see Figure \ref{auto3}). At state 5 the parser reduces $id_2$ to $F$.\\ \\
At the same state, CLR parser would not reduce it and would straight away fire an error. This is because the CLR parser contains a set of lookahead symbols for each item which would not contain ']' for that reduction at that state. At that state the item would like $\{F \rightarrow id. \tab \tab ),+,*\}$, because it has already seen a '(' and expects '),+,*' to reduce $id_2$.\\ \\
\textbf{(c).} Consider the same example string as above. Now, $\{F \rightarrow id \tab \tab ],+,*\}$, must also be an item  at some other state in the CLR parser. This is because the CLR parser accepts $id[id] := id $. \\

But the LALR parser merges these two states of the CLR parser to get an item like $\{F \rightarrow id \tab \tab ],),*,+ \}$. Now, similar to SLR parser LALR parser also reduces $id_2$ to $F$ immediately since now ']' is in the set of lookahead symbols. But, as we saw earlier, that CLR parser would throw an error.
\end{document}	



