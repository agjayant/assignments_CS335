\documentclass{article}
\usepackage{geometry}
\usepackage{graphicx}
\usepackage{amsmath}
\usepackage{algorithm}
\usepackage{algpseudocode}
\usepackage{multicol}
\usepackage{multirow}
\usepackage{arydshln}

\geometry{
a4paper,
right=10mm,
left=10mm,
top=10mm,
bottom=10mm,	
}
\newcommand\tab[1][5pt]{\hspace*{#1}} 
\begin{document}

\pagenumbering{gobble}

\begin{center}
\textbf{\huge Assignment 2 : CS335} \\
\textit{\Large Jayant Agrawal}         14282 \\
Section S1
\end{center}

\textbf{\huge Parsing}

\section*{Sol2: Parse Table}
The grammer: $G$ has the following productions(numbered):
\begin{enumerate}
\item $S \rightarrow Ma$
\item $S \rightarrow bMc$
\item $S \rightarrow dc$
\item $S \rightarrow bda$
\item $M \rightarrow d$
\end{enumerate}
Adding the production $S' \rightarrow S$ to get augmented grammer $G'$ and computing \emph{Sets-of-items}:
\begin{multicols}{2}
\begin{equation*}
\begin{aligned}
I_0 &: S' \rightarrow .S  \\
&: S \rightarrow .Ma \hspace{5pt} | \tab .bMc \tab | \tab .dc \tab | \tab .bda \\
&: M \rightarrow .d
\end{aligned}
\end{equation*}

\begin{equation*}
\begin{aligned}
I_1 : goto(I_0,S) &: S' \rightarrow S.  \\
\end{aligned}
\end{equation*}


\begin{equation*}
\begin{aligned}
I_2 : goto(I_0,M) &: S \rightarrow M.a  \\
\end{aligned}
\end{equation*}


\begin{equation*}
\begin{aligned}
I_3 : goto(I_2,a) &: S \rightarrow Ma.  \\
\end{aligned}
\end{equation*}


\begin{equation*}
\begin{aligned}
I_4 : goto(I_0,b) &: S \rightarrow b.Mc \tab | \tab b.da  \\
&: M \rightarrow .d \\
\end{aligned}
\end{equation*}


\begin{equation*}
\begin{aligned}
I_5 : goto(I_4,M) &: S \rightarrow bM.c  \\
\end{aligned}
\end{equation*}


\begin{equation*}
\begin{aligned}
I_6 : goto(I_4,d) &: S \rightarrow bd.a  \\
&: M \rightarrow d.  \\
\end{aligned}
\end{equation*}

\begin{equation*}
\begin{aligned}
I_7 : goto(I_5,c) &: S \rightarrow bMc.  \\
\end{aligned}
\end{equation*}

\begin{equation*}
\begin{aligned}
I_8 : goto(I_6,a) &: S \rightarrow bda.  \\
\end{aligned}
\end{equation*}

\begin{equation*}
\begin{aligned}
I_9 : goto(I_0,d) &: S \rightarrow d.c  \\
&: M \rightarrow d.  \\
\end{aligned}
\end{equation*}


\begin{equation*}
\begin{aligned}
I_{10} : goto(I_9,c) &: S \rightarrow dc.  \\
\end{aligned}
\end{equation*}
\end{multicols}

Also, the first and follow sets of the non-terminals in $G$ are as follows:\\
first(S) = \{b,d\}\\
first(M) = \{d\} \\
follow(S) = \{\$\} \\
follow(M) = \{a,c\} \\ \\
Using the above first, follow sets and the sets-of-items, we get the following SLR Parse Table :
\begin{center}
\begin{tabular}{||c | c c c c c | c c||}
\hline
\hline
\multirow{2}{*}{\textbf{STATE}} & \multicolumn{5}{c|}{\emph{action}}  & \multicolumn{2}{c||}{\emph{goto}} \\
\cline{2-8}
& \bf a & \bf b & \bf c & \bf d & \$ & \bf S &\bf M \\
\hline
0 & & s4 & & s9 & & 1 &2 \\
\hdashline[0.5pt/10pt]
1 & & & & & \emph{acc} &  & \\
\hdashline[0.5pt/10pt]
2 &s3 &  & &  & &  & \\
\hdashline[0.5pt/10pt]
3 & & & & & r1 &  & \\
\hdashline[0.5pt/10pt]
4 & & & & s6 &  &  & 5\\
\hdashline[0.5pt/10pt]
5 & & & s7& &  &  & \\
\hdashline[0.5pt/10pt]
6 & \begin{tabular}{c} s8 \\ r5 \end{tabular} & & r5& &  &  & \\
\hdashline[0.5pt/10pt]
7 &  & & & & r2 &  & \\
\hdashline[0.5pt/10pt]
8 &  & & & & r4 &  & \\
\hdashline[0.5pt/10pt]
9 & r5 & &\begin{tabular}{c} s10 \\ r5 \end{tabular} & &  &  & \\
\hdashline[0.5pt/10pt]
10 &  & & & & r3 &  & \\
\hline
\hline
\end{tabular}
\end{center}

\end{document}



